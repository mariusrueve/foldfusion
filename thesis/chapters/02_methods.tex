\Chapter{Methods}
\label{chap:methods}

\section{Overview}
The framework implements a modular pipeline for systematic augmentation and assessment of predicted protein structures using experimentally resolved homologous complexes and associated ligands. The architecture consists of (i)

The framework implements a modular pipeline for systematic augmentation and assessment of predicted protein structures using experimentally resolved homologous complexes and associated ligands. Its architecture comprises (i) structured acquisition of target and donor structural data, (ii) detection and characterization of binding environments, (iii) sequence/structure alignment and geometric mapping, (iv) ligand and pocket feature transplantation with conflict resolution, (v) multi-criteria filtering and scoring, and (vi) standardized evaluation and visualization of resulting models and annotations. This layered decomposition promotes separation of concerns, controlled extensibility, and reproducible execution.

Inputs consist of one or more target protein identifiers (e.g., UniProt accessions) whose corresponding predicted three-dimensional models are retrieved (or validated if cached) and normalized. Donor candidates are gathered from curated structural repositories and homolog databases according to configurable similarity, coverage, and quality constraints. Auxiliary metadata (ligands, binding site residues, cavity descriptors, and experimental provenance) are harvested through dedicated tool adapters. Each external data source is encapsulated behind a dedicated integration component to isolate protocol specifics from core orchestration logic.

Alignment and transplantation proceed via staged refinement: coarse-grained sequence or domain-level correspondence, optional structural superposition, residue-level mapping, and coordinate transfer for ligand entities and binding site annotations. Conflict handling addresses steric clashes, incomplete residue definitions, alternate conformers, and chain indexing inconsistencies. A filtering and scoring layer then applies geometric, physicochemical, and provenance-derived criteria (e.g., clash metrics, distance thresholds, ligand completeness, resolution provenance proxies) to retain only high-confidence transplanted assemblies. Scoring outputs are structured to permit downstream comparative analyses across targets, donors, and parameter settings.

Evaluation utilities quantify improvement and reliability through metrics such as structural alignment quality, pocket conservation, ligand retention integrity, and scoring distribution profiles. Visualization components generate comparative plots and summary artifacts to support interpretability and methodological diagnosis. Configuration management and structured logging ensure parameter traceability, deterministic re-runs, and auditability of intermediate decisions. The overall design facilitates insertion of new data sources, scoring strategies, or analytical endpoints with minimal impact on existing workflow stages.

\begin{figure}[H]
    \centering
    % Placeholder: replace with actual pipeline diagram (e.g., TikZ or imported PDF)
    \fbox{\rule[0pt]{0pt}{4cm} Pipeline flow diagram placeholder}
    \caption{High-level pipeline stages: data acquisition; binding site and ligand characterization; alignment and geometric mapping; transplantation; filtering and scoring; evaluation and visualization.}
    \label{fig:pipeline_overview}
\end{figure}


\section{Data and Inputs}
\Todo{Describe the input protein models (AlphaFold/UniProt IDs), how they are fetched, and the benchmark dataset in evaluation/data (size, selection criteria). Include table summarizing proteins used.}
\Todo{Specify configuration via config.toml: key parameters (alignment thresholds, filters), seeds, and runtime knobs.}

\section{Components and Tools}
\subsection{AlphaFold Fetcher}
\Todo{Explain how predicted structures are obtained: AlphaFold DB URLs, version fallback v4\textrightarrow{}v3, file formats (PDB), and caching/normalization strategy.}
\Todo{Detail preprocessing: pLDDT-based trimming of unreliable segments (\ensuremath{\ge}5 consecutive residues with pLDDT $<$ 50) as implemented in the pipeline; justify threshold and segment-based approach versus per-residue masking.}
\Todo{Note chain indexing/altloc handling from AlphaFold PDBs, and any renaming or path conventions used in the output directory structure (\texttt{AlphaFold/}).}

\subsection{Binding Site Prediction (DoGSite3)}
\Todo{Summarize DoGSite3's purpose (pocket detection via Difference of Gaussian on protein surface; physicochemical descriptors). Include citation and tool version.}
\Todo{Describe invocation used in the pipeline: \texttt{--writeSiteResiduesEDF}, input PDB from AlphaFold, output directory structure (\texttt{Dogsite3/}).}
\Todo{State selection of the top-ranked pocket EDF (\texttt{output\_P\_1\_res.edf}) and the REFERENCE fix-up to the absolute input PDB path performed post-run.}
\Todo{List key pocket features exported in EDF that downstream tools (SIENA) consume.}

\subsection{Donor Databases and Retrieval (PDB/UniProt/SIENA DB)}
\Todo{Detail sources of donor structures and ligands: PDB (version/date), UniProt mapping, and SIENA database preparation path. Include database versions and citations.}
\Todo{Explain how the SIENA database is generated/initialized once per run (input PDB directory, format), where it is stored, and when regeneration is skipped.}
\Todo{State any filtering on donors prior to alignment (resolution, polymer type, ligand presence).}

\subsection{Binding-Site Similarity Search and Alignment (SIENA)}
\Todo{Describe inputs (EDF pocket from DoGSite3) and the SIENA search/alignment workflow; include citation and tool version.}
\Todo{Document command parameters used: \texttt{--edf}, \texttt{--database}, \texttt{--output .}, identity cutoff (\texttt{--identity 0.85}); justify chosen thresholds.}
\Todo{Explain result parsing: reading \texttt{resultStatistic.csv} (semicolon separated), column cleanup, sorting primarily by Active site identity (desc), then Backbone and All-atom RMSD (asc).}
\Todo{Clarify use of \texttt{ensemble/\*.pdb} files for downstream ligand extraction/transplantation and how many top alignments are retained (configurable \texttt{siena\_max\_alignments}).}

\subsection{Ligand Optimization and Scoring (JAMDA)}
\Todo{Summarize JAMDA's role: energy-based pose optimization/scoring of transplanted ligands; include citation and tool version.}
\Todo{Describe invocation pattern per ligand: inputs (AlphaFold PDB, ligand SDF), outputs (optimized SDF), and flags used (\texttt{--optimize}); output organization (\texttt{JamdaScorer/<PDB>/ligand\_id.sdf}).}
\Todo{State what scores/outputs are captured and how they inform filtering (e.g., clash reduction, score improvements). Cross-reference concrete thresholds in Filtering and Scoring.}
\subsection{Alignment}
\Todo{Describe global/local alignment methods used; define RMSD/score variants and parameters.}
\subsection{Transplantation}
\Todo{Explain coordinate mapping, handling of alternate locations, protonation states, and chain/residue mapping.}
\subsection{Filtering and Scoring}
\Todo{Define clash checks, distance cutoffs, and quality indicators; justify chosen thresholds. Reference how SIENA metrics (identity, RMSDs) and JAMDA outputs contribute to accept/reject decisions.}

\section{Implementation Details}
\Todo{Summarize code structure (packages in foldfusion/), logging, CLI entry points, and dependency management (pyproject/uv.lock).}
\Todo{Discuss error handling and retries for external resources; note offline/cached mode.}

\section{Reproducibility}
\Todo{Document exact environment (Python version, key packages), compute hardware, and commands to reproduce results. Reference Appendix for full config and commit hash.}

\section{Ethical and Licensing Considerations}
\Todo{Note database/tool licenses and usage restrictions; acknowledge limitations of homology-based inference.}
